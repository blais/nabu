\title[Nabu]{Nabu: A Publishing System Using (Re) Structured Text}

%-------------------------------------------------------------------------------
\begin{frame}
  \frametitle{Motivation}

slide with the problem to solve


\end{frame}


- I write lots of little text files

  ... example ...

  - Configuration notes

    -> as these notes

  - Timesheet



What is Nabu?
=============

- show diagram




Example Extractor
=================

- go through an example of writing an extractor





Applications
============

- blog

  - selective delivery of information

    e.g. ``soft privileges'', a link that expires, that allows access to a
    specific resource, e.g. I want to share my image gallery with my boss

- calendar
- timesheet

- ...


Advantages
==========

- Any document structures can be reused and combined to provide a rich content



Downsides
=========

- You have to understand how restructuredtext gets parsed

  - You end up getting dragged into the docutils project a bit
  
  --> You can use rst2pseudoxml.py to easily debug and learn input syntaxes


Future Work
===========

- Support encryption in the publisher
- Support per-document options (... adds complexity)


Conclusion
==========


- uses: imagine a static pages site, but you want to update the special events
  yourself (the client pays for this).  You can do that with a simple text file
  and Nabu.





========================================================
  Strike Two -- Ideas for the talk from Rachel's place
========================================================

1992 - Bookmarks
================

I enter university

- using Xmosaic
- bookmarking sites left and right

...bmmgr was born

Outputs to Xmosaic, Netscape, Mozilla, IE, Firefox, ... and what next?

- I would like to share some of these bookmarks
- Searching

del.ici.us


1997 - Address Book
===================

- in M.Sc., using Netscape to store my contacts
- my director, a nroff/refer shows me his simple ascii system for his address
  book

... been living happily with a text file since them.

- Using pargrep.pl


1999 - Blog
===========

- friend Patrick from Synaptic does one of the early travelogues on the internet
- a blog

... adventures script is born.  One of my first Python programs.

- much content served
- input in XML format
- need to regenerate the static pages everytime


200?
====

- I discover restructured text

2004 - Technical Text Files
===========================

- I'm getting a little bit old, I'm losing memory
- Now I know that I'm stupid and that I will forget

... everytime I start a new task, I start a new text file to take notes

- I can grep the files
- I often want to share many of these short technical documents



2005 - Travel Files - the Appearance of Mixed Data
==================================================

.. <diagram with files and information stored across them>

What if I could identify and extract the meaningful parts of information from
those files and store it appropriately?

What could I build with this?

- You may have heard this idea before: the Semantic Web

  It would allow Google to search the "address book" of the entire internet!

- Many years in the making, will probably never happen in its ideal form


The Goal
========

  Build a system that can extract these informations in my own selfish set of
  personal files and store them in a "semantic" way, so I can use this
  information and serve in more interesting ways later (in database tables).


Target Audience
===============

Not for your mom!

- Programmers have developed this great ability: to edit text files.

  - we understand indentation
  - we know about spacing
  - lines of separation, justification, filling, etc.

  Leverage it!


I live in Emacs
===============

When I first log in on my machine, I invariably do

1. start a shell
2. start emacs
3. start a web browser

Most of you are probably doing the same.

Emacs or vi are always kept running.


Wikis Suck
==========

Why?

- Anything but the most trivial document title is a nightmare

  BrazilTravelNotes

- The editor facilities in most of the browsers is lame beyond boundaries.

... Cool idea: link an emacs instance within Firefox.

- Does not identify the meanings within the files either.


The Dependency Problem
======================

More dependencies means

- it's harder to install
- it's usable on less platforms
- the lifetime of the project is linked with the lifetime of the shorter
  dependency 

On the **client**, we depend only on:

- Python
- Text files (i.e. files with one of the predefined encodings)



ReStructuredText
================

.. ask Goodger to get up

   "If you see this guy in the corridor, show him some love, give him a hug."

- The best text-to-structure conversion tool.
- Finds the best compromises (all decisions are documented thoroughly)


.. <show some text with recursive boxes in it> 





Desktop Search
==============

Imagine if you could use your desktop database to feed your blog.


